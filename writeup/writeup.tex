\documentclass[a4paper,12pt]{article}
\usepackage{geometry}
\usepackage{hyperref}
\usepackage{multicol}
\usepackage{multirow}
\usepackage{graphicx}
\usepackage{booktabs}
\usepackage{tabularx}
\usepackage{enumitem}
\usepackage{longtable}
\usepackage{tabularray}

\geometry{top=1in, bottom=1in, left=1in, right=1in}

\begin{document}

% Title Block (No Forced Page Break)
\noindent
{\LARGE \textbf{Execution Plan for CVWO Assignment}} \\[12pt]
{\large Yu Letian} \\[6pt]
{\normalsize December 27, 2024} \\[20pt]

\begin{multicols}{2}  % Start 2-column layout

\section{Objective}
The goal is to build an online web forum with the following tech stack:
\begin{itemize}
    \item \textbf{Frontend:} React.js + TypeScript
    \item \textbf{Backend:} Ruby on Rails
    \item \textbf{Database:} SQLite
\end{itemize}

\noindent The forum must include:
\begin{itemize}
    \item An authentication system
    \item Create, read, update, and delete (CRUD) operations for forum threads and comments
    \item A tagging system
\end{itemize}

\section{User Requirements}
\textbf{User Stories}
\begin{itemize}
    \item As a user, I want an authentication system so that I can distinguish different threads and comments made by different people.
    \item As a user, I want to create, read, update, and delete forum threads and comments (the basic functionality of a web forum).
    \item As a user, I want a tagging system so I can find topics aligned with my interests.
\end{itemize}

To be more specific, the web forum will utilize the following HTTP methods to meet the CRUD requirements:
\begin{itemize}
    \item \textbf{POST} for creating new threads and comments
    \item \textbf{GET} for reading threads and comments
    \item \textbf{PUT} for updating existing threads or comments
    \item \textbf{DELETE} for deleting threads or comments
\end{itemize}

\vspace{10pt}

Data includes each thread or comment’s creation time, modification time, content, tagging, and author (by username). Every thread will have a unique ID (e.g., T1, T2, \dots) and every comment will have a unique ID (e.g., C1, C2, \dots). These IDs serve as primary keys in the database.

\vspace{10pt}

There should be two pages:
\begin{itemize}
    \item A \textbf{login page}, where users authenticate themselves
    \item A \textbf{main forum page}, containing thread listings and styled with HTML/CSS
\end{itemize}

\begin{longtblr}[
  caption={Use Case}
]{p{2cm}|p{5cm}}
\textbf{Use Case} & \textbf{Details} \\ \hline
Primary actor & Users who want to communicate online \\
Pre-condition & The user has already authenticated by username \\
Success end condition & A window appears indicating the thread/comment is successfully created or updated \\
Failed end condition & A window appears indicating an error occurred \\
Trigger & Clicking the new thread/comment button \\
Open issue & What if two users use the same username?  
            \newline \textit{Proposed fix:} add a postfix number (1, 2, \dots) to distinguish identical usernames \\
Main process &
\begin{minipage}[t]{\linewidth}
    \begin{enumerate}[leftmargin=1em]
        \item The user enters his/her username.
        \item The user clicks "New Thread" to open a new window, types in a title, content, and creates a tag for it by using the symbol `\#`.
        \item The user searches in the search box by tag. Tags are separated with the symbol `\#`.
        \item The user can view a thread by clicking on it.
        \item The user can comment by clicking the "Comment" button, but cannot comment on a comment.
    \end{enumerate}
\end{minipage} \\
Details &
\begin{minipage}[t]{\linewidth}
    \begin{enumerate}[leftmargin=1em]
        \item \textbf{(1):} Check if the user is already signed up. 
              \newline If so, authenticate the user. 
              \newline If not, add a new username to the database.
        \item \textbf{(4):} Given a user-input tag, the application searches the database to see if the tag exists, then finds all threads that match the required tags.
    \end{enumerate}
\end{minipage} \\
\end{longtblr}

\section{Learning Approach}
As a newcomer to web development, I plan to invest significant time in:
\begin{itemize}[leftmargin=1em]
    \item \textbf{Fundamentals:}  
        Understanding how servers, databases, and frontends interact. by going through \href{https://www.theodinproject.com/paths/foundations/courses/foundations}{The Odin Project}. in reflection, this actually take longer time than anticipated.
    \item \textbf{Framework Proficiency:}  
        Exploring Ruby on Rails for backend logic, and React.js + TypeScript for frontend components.  
    \item \textbf{Practice and Examples:}  
        Play around with the skeleton project, building smaller parts, and experimenting with CRUD functionalities.  
    \item \textbf{Debugging \& Testing:}  
        Learning how to troubleshoot issues, write tests, and ensure code reliability.  
    \item \textbf{Documentation:}  
        Reading official docs and community resources to stay informed and solve problems efficiently.
\end{itemize}

\section{Implementation Outline}
To manage the overall development process, I will:
\begin{itemize}
    \item Initialize the project with Rails and set up the database schema for threads and comments.
    \item Create user authentication mechanisms (sign-up, login, logout) with secure password handling.
    \item Develop the frontend in React + TypeScript, integrating with Rails through RESTful APIs.
    \item Add tagging logic and searching functionality for a smoother user experience.
    \item Continuously test functionality using simple unit tests.
\end{itemize}

\noindent final modified at January 4, 2025 \\ by Yu Letian

\end{multicols}
\end{document}